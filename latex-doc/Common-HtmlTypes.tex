\haddockmoduleheading{Common.HtmlTypes}
\label{module:Common.HtmlTypes}
\haddockbeginheader
{\haddockverb\begin{verbatim}
module Common.HtmlTypes (
    Html(Html),  Head(Head),  Body,  BlockElem(A, Hr, Table, P),  Tr(Tr),  Td, 
    Title,  HtmlAlgebra,  foldHtml
  ) where\end{verbatim}}
\haddockendheader

This module contains our representation of an HTML document. Our 
 simplistic version of HTML only knows anchors, horizontal rules, tables and paragraphs, 
 but this is enough to display a BibTeX database. 
\par

\begin{haddockdesc}
\item[\begin{tabular}{@{}l}
data\ Html
\end{tabular}]\haddockbegindoc
\haddockbeginconstrs
\haddockdecltt{=} & \haddockdecltt{Html Head Body} & \\
\end{tabulary}\par
an HTML document has a head and a body
\par

\end{haddockdesc}
\begin{haddockdesc}
\item[\begin{tabular}{@{}l}
instance\ Show\ Html\\instance\ Tree\ Html
\end{tabular}]
\end{haddockdesc}
\begin{haddockdesc}
\item[\begin{tabular}{@{}l}
data\ Head
\end{tabular}]\haddockbegindoc
\haddockbeginconstrs
\haddockdecltt{=} & \haddockdecltt{Head Title} & \\
\end{tabulary}\par
the head only contains the title.
\par

\end{haddockdesc}
\begin{haddockdesc}
\item[\begin{tabular}{@{}l}
instance\ Show\ Head\\instance\ Tree\ Head
\end{tabular}]
\end{haddockdesc}
\begin{haddockdesc}
\item[\begin{tabular}{@{}l}
type\ Body\ =\ {\char 91}BlockElem{\char 93}
\end{tabular}]\haddockbegindoc
the body of an HTML document is a list of block elements (table, anchor, etc.)
\par

\end{haddockdesc}
\begin{haddockdesc}
\item[\begin{tabular}{@{}l}
data\ BlockElem
\end{tabular}]\haddockbegindoc
\haddockbeginconstrs
\haddockdecltt{=} & \haddockdecltt{A [Field] Reference} & \\
\haddockdecltt{|} & \haddockdecltt{Hr} & \\
\haddockdecltt{|} & \haddockdecltt{Table [Field] [Tr]} & \\
\haddockdecltt{|} & \haddockdecltt{P [Field] String} & \\
\end{tabulary}\par
a block element can be (as stated) anchor, rule, table or a paragraph (a paragraph
 is also used to represent a plain string. In this case it's attribute list is empty, 
 and when rendering, the p-tag is ommitted.
\par

\end{haddockdesc}
\begin{haddockdesc}
\item[\begin{tabular}{@{}l}
instance\ Show\ BlockElem\\instance\ Tree\ BlockElem
\end{tabular}]
\end{haddockdesc}
\begin{haddockdesc}
\item[\begin{tabular}{@{}l}
data\ Tr
\end{tabular}]\haddockbegindoc
\haddockbeginconstrs
\haddockdecltt{=} & \haddockdecltt{Tr [Field] [Td]} & \\
\end{tabulary}\par
a table row. It has attributes and a list of cells
\par

\end{haddockdesc}
\begin{haddockdesc}
\item[\begin{tabular}{@{}l}
instance\ Show\ Tr\\instance\ Tree\ Tr
\end{tabular}]
\end{haddockdesc}
\begin{haddockdesc}
\item[\begin{tabular}{@{}l}
type\ Td\ =\ BlockElem
\end{tabular}]\haddockbegindoc
a table cell is a block element
\par

\end{haddockdesc}
\begin{haddockdesc}
\item[\begin{tabular}{@{}l}
type\ Title\ =\ String
\end{tabular}]\haddockbegindoc
the title of a document is simply a string. 
\par

\end{haddockdesc}
\begin{haddockdesc}
\item[\begin{tabular}{@{}l}
type\ HtmlAlgebra\ html\ head\ body\ block\ tr\ =\ (head\\\ \ \ \ \ \ \ \ \ \ \ \ \ \ \ \ \ \ \ \ \ \ \ \ \ \ \ \ \ \ \ \ \ \ \ \ \ \ \ \ \ \ \ \ ->\ body\ ->\ html,\ (Title\ ->\ head,\ {\char 91}block{\char 93}\\\ \ \ \ \ \ \ \ \ \ \ \ \ \ \ \ \ \ \ \ \ \ \ \ \ \ \ \ \ \ \ \ \ \ \ \ \ \ \ \ \ \ \ \ \ \ \ \ \ \ \ \ \ \ \ \ \ \ \ \ \ \ \ \ \ \ \ \ \ \ \ \ \ \ \ \ \ ->\ body),\ ({\char 91}Field{\char 93}\\\ \ \ \ \ \ \ \ \ \ \ \ \ \ \ \ \ \ \ \ \ \ \ \ \ \ \ \ \ \ \ \ \ \ \ \ \ \ \ \ \ \ \ \ \ \ \ \ \ \ \ \ \ \ \ \ \ \ \ \ \ \ \ \ \ \ \ \ \ \ \ \ \ \ \ \ \ \ \ \ \ \ \ \ \ \ \ \ ->\ Reference\\\ \ \ \ \ \ \ \ \ \ \ \ \ \ \ \ \ \ \ \ \ \ \ \ \ \ \ \ \ \ \ \ \ \ \ \ \ \ \ \ \ \ \ \ \ \ \ \ \ \ \ \ \ \ \ \ \ \ \ \ \ \ \ \ \ \ \ \ \ \ \ \ \ \ \ \ \ \ \ \ \ \ \ \ \ \ \ \ \ \ \ ->\ block,\ block,\ {\char 91}Field{\char 93}\\\ \ \ \ \ \ \ \ \ \ \ \ \ \ \ \ \ \ \ \ \ \ \ \ \ \ \ \ \ \ \ \ \ \ \ \ \ \ \ \ \ \ \ \ \ \ \ \ \ \ \ \ \ \ \ \ \ \ \ \ \ \ \ \ \ \ \ \ \ \ \ \ \ \ \ \ \ \ \ \ \ \ \ \ \ \ \ \ \ \ \ \ \ \ \ \ \ \ \ \ \ \ \ \ \ \ \ \ ->\ {\char 91}tr{\char 93}\\\ \ \ \ \ \ \ \ \ \ \ \ \ \ \ \ \ \ \ \ \ \ \ \ \ \ \ \ \ \ \ \ \ \ \ \ \ \ \ \ \ \ \ \ \ \ \ \ \ \ \ \ \ \ \ \ \ \ \ \ \ \ \ \ \ \ \ \ \ \ \ \ \ \ \ \ \ \ \ \ \ \ \ \ \ \ \ \ \ \ \ \ \ \ \ \ \ \ \ \ \ \ \ \ \ \ \ \ \ \ \ ->\ block,\ {\char 91}Field{\char 93}\\\ \ \ \ \ \ \ \ \ \ \ \ \ \ \ \ \ \ \ \ \ \ \ \ \ \ \ \ \ \ \ \ \ \ \ \ \ \ \ \ \ \ \ \ \ \ \ \ \ \ \ \ \ \ \ \ \ \ \ \ \ \ \ \ \ \ \ \ \ \ \ \ \ \ \ \ \ \ \ \ \ \ \ \ \ \ \ \ \ \ \ \ \ \ \ \ \ \ \ \ \ \ \ \ \ \ \ \ \ \ \ \ \ \ \ \ \ \ \ \ \ ->\ String\\\ \ \ \ \ \ \ \ \ \ \ \ \ \ \ \ \ \ \ \ \ \ \ \ \ \ \ \ \ \ \ \ \ \ \ \ \ \ \ \ \ \ \ \ \ \ \ \ \ \ \ \ \ \ \ \ \ \ \ \ \ \ \ \ \ \ \ \ \ \ \ \ \ \ \ \ \ \ \ \ \ \ \ \ \ \ \ \ \ \ \ \ \ \ \ \ \ \ \ \ \ \ \ \ \ \ \ \ \ \ \ \ \ \ \ \ \ \ \ \ \ \ \ \ ->\ block),\ {\char 91}Field{\char 93}\\\ \ \ \ \ \ \ \ \ \ \ \ \ \ \ \ \ \ \ \ \ \ \ \ \ \ \ \ \ \ \ \ \ \ \ \ \ \ \ \ \ \ \ \ \ \ \ \ \ \ \ \ \ \ \ \ \ \ \ \ \ \ \ \ \ \ \ \ \ \ \ \ \ \ \ \ \ \ \ \ \ \ \ \ \ \ \ \ \ \ \ \ \ \ \ \ \ \ \ \ \ \ \ \ \ \ \ \ \ \ \ \ \ \ \ \ \ \ \ \ \ \ \ \ \ \ \ \ \ \ \ \ \ \ \ ->\ {\char 91}block{\char 93}\\\ \ \ \ \ \ \ \ \ \ \ \ \ \ \ \ \ \ \ \ \ \ \ \ \ \ \ \ \ \ \ \ \ \ \ \ \ \ \ \ \ \ \ \ \ \ \ \ \ \ \ \ \ \ \ \ \ \ \ \ \ \ \ \ \ \ \ \ \ \ \ \ \ \ \ \ \ \ \ \ \ \ \ \ \ \ \ \ \ \ \ \ \ \ \ \ \ \ \ \ \ \ \ \ \ \ \ \ \ \ \ \ \ \ \ \ \ \ \ \ \ \ \ \ \ \ \ \ \ \ \ \ \ \ \ \ \ \ ->\ tr)
\end{tabular}]\haddockbegindoc
And once again we define an algebra for folding over an HTML document.
 This will prove useful when we want to pretty-print the html (or render
 it in any form), since the above type is only an abstract representation of 
 an HTML document.
\par
Since Html is tree-like, it seems logical to use a fold to convert it to some other
 (tree-like) format, in our case, this is Doc, the CCO pretty-print data structure. The actual
 fold for this is defined in PrettyPrintHTML.Tool. 
\par

\end{haddockdesc}
\begin{haddockdesc}
\item[\begin{tabular}{@{}l}
foldHtml\ ::\ HtmlAlgebra\ html\ head\ body\ block\ tr\ ->\ Html\ ->\ html
\end{tabular}]\haddockbegindoc
The function which folds over an HTML tree, given an algebra as defined above.
\par

\end{haddockdesc}