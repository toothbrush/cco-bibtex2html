\haddockmoduleheading{AbstractBib2HTML.Tool}
\label{module:AbstractBib2HTML.Tool}
\haddockbeginheader
{\haddockverb\begin{verbatim}
module AbstractBib2HTML.Tool (
    mainFunc,  pipeline,  checkRequired,  empty,  sortFields,  checkOptionals, 
    checkEntry,  sorter,  sortGen,  checkDups,  bibTex2HTML,  bib2htmlAlg, 
    separate,  generateIndex,  generateTableRows,  flattenEntry,  formatFields
  ) where\end{verbatim}}
\haddockendheader

this module implements the bib2html tool. It contains the complete pipeline for
 this part of the suite, converting from a BibTex ATerm into an Html ATerm.
 This is also where the main validation is done on the BibTeX database, to make sure that there
 are no duplicates, etc. 
\par

\begin{haddockdesc}
\item[\begin{tabular}{@{}l}
mainFunc\ ::\ IO\ ()
\end{tabular}]\haddockbegindoc
the default main function. Just wrap a pipeline in IO
\par

\end{haddockdesc}
\begin{haddockdesc}
\item[\begin{tabular}{@{}l}
pipeline\ ::\ Component\ String\ String
\end{tabular}]\haddockbegindoc
The pipeline for bib2html. 
\par

\end{haddockdesc}
\begin{haddockdesc}
\item[\begin{tabular}{@{}l}
checkRequired\ ::\ Component\ BibTex\ BibTex
\end{tabular}]\haddockbegindoc
checks a number of things. Make sure all required fields are in all entries, and that there are no fields 
 which are unrecognised (such as misspelled fields). Also sorts the fields (author, title, the rest, finally year). Finally
 checks that all the other fields are at least optional, since we don't want disallowed fields in entries. 
\par

\end{haddockdesc}
\begin{haddockdesc}
\item[\begin{tabular}{@{}l}
empty\ ::\ Entry\ ->\ Bool
\end{tabular}]\haddockbegindoc
this function returns whether a given entry has 0 fields. We don't want those. 
\par

\end{haddockdesc}
\begin{haddockdesc}
\item[\begin{tabular}{@{}l}
sortFields\ ::\ Entry\ ->\ Feedback\ Entry
\end{tabular}]\haddockbegindoc
Sort fields, and nub duplicates. Here we just need to call \haddockid{sort}, since
 the Ord class is implemented on Fields.
\par

\end{haddockdesc}
\begin{haddockdesc}
\item[\begin{tabular}{@{}l}
checkOptionals\ ::\ Entry\ ->\ Feedback\ Entry
\end{tabular}]\haddockbegindoc
Checks that all fields in an entry are at least optional. If they are not, issue a warning and empty them. These will later be pruned.
\par

\end{haddockdesc}
\begin{haddockdesc}
\item[\begin{tabular}{@{}l}
checkEntry\ ::\ Entry\ ->\ Feedback\ Entry
\end{tabular}]\haddockbegindoc
Check that all required fields are present, depending on the entry's type. If they aren't issue an error and stop. This is the only 
 condition on which the bib2html program fails. 
\par

\end{haddockdesc}
\begin{haddockdesc}
\item[\begin{tabular}{@{}l}
sorter\ ::\ Component\ BibTex\ BibTex
\end{tabular}]\haddockbegindoc
Sort entries by author, year, title. Makes use of the usual lexical sort on strings. 
\par

\end{haddockdesc}
\begin{haddockdesc}
\item[\begin{tabular}{@{}l}
sortGen
\end{tabular}]\haddockbegindoc
\haddockbeginargs
\haddockdecltt{::} & \haddockdecltt{String} & the key to sort on
 \\
                                              \haddockdecltt{->} & \haddockdecltt{Entry} & entry 1
 \\
                                                                                           \haddockdecltt{->} & \haddockdecltt{Entry} & entry 2
 \\
                                                                                                                                        \haddockdecltt{->} & \haddockdecltt{Ordering} & \\
\end{tabulary}\par
helper function which compares two entries and returns an ordering, based on the key requested.
\par

\end{haddockdesc}
\begin{haddockdesc}
\item[\begin{tabular}{@{}l}
checkDups\ ::\ Component\ BibTex\ BibTex
\end{tabular}]\haddockbegindoc
This function eliminates entries with the same name. The first found entry with a certain name is retained. 
 A warning is also issued. 
\par

\end{haddockdesc}
\begin{haddockdesc}
\item[\begin{tabular}{@{}l}
bibTex2HTML\ ::\ Component\ BibTex\ Html
\end{tabular}]\haddockbegindoc
Convert a BibTex tree to Html by folding with the bib2htmlAlg algebra.
\par

\end{haddockdesc}
\begin{haddockdesc}
\item[\begin{tabular}{@{}l}
bib2htmlAlg\ ::\ BibTexAlgebra\ Html\ BlockElem\ ({\char 91}BlockElem{\char 93},\ Tr)
\end{tabular}]\haddockbegindoc
this algebra converts a BibTex tree into an Html representation, including
 a list of hyperlinks at the top, maybe preamble blocks, and the table with the
 entries. 
\par

\end{haddockdesc}
\begin{haddockdesc}
\item[\begin{tabular}{@{}l}
separate\ ::\ {\char 91}BlockElem{\char 93}\ ->\ {\char 91}BlockElem{\char 93}
\end{tabular}]\haddockbegindoc
used to place a pipe character between the list of hyperlinks 
\par

\end{haddockdesc}
\begin{haddockdesc}
\item[\begin{tabular}{@{}l}
generateIndex\ ::\ Reference\ ->\ {\char 91}BlockElem{\char 93}
\end{tabular}]\haddockbegindoc
turns a reference into a hyperlink 
\par

\end{haddockdesc}
\begin{haddockdesc}
\item[\begin{tabular}{@{}l}
generateTableRows\ ::\ EntryType\ ->\ Reference\ ->\ {\char 91}Field{\char 93}\ ->\ Tr
\end{tabular}]\haddockbegindoc
turns an Entry (or more accurately, given a type, a name, and a list of attributes) into
 a table row
\par

\end{haddockdesc}
\begin{haddockdesc}
\item[\begin{tabular}{@{}l}
flattenEntry\ ::\ EntryType\ ->\ {\char 91}Field{\char 93}\ ->\ String
\end{tabular}]\haddockbegindoc
given a type and a list of attributes, flattens an entry into a string, for placement
 in a table cell
\par

\end{haddockdesc}
\begin{haddockdesc}
\item[\begin{tabular}{@{}l}
formatFields\ ::\ Field\ ->\ String
\end{tabular}]\haddockbegindoc
format a field into a string. Possibly do special formatting things, depending on 
 if it's a title, for example. 
\par

\end{haddockdesc}