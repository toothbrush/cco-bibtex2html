\haddockmoduleheading{PrettyPrintHTML.Tool}
\label{module:PrettyPrintHTML.Tool}
\haddockbeginheader
{\haddockverb\begin{verbatim}
module PrettyPrintHTML.Tool (
    mainFunc,  pipeline,  aTerm2Html,  html2String,  html2ppAlgebra, 
    tagWithFields,  printField
  ) where\end{verbatim}}
\haddockendheader

This module contains the pp-html tool's code. It is a rather simple tool: it 
 reads an Html ATerm, folds over the Html to convert it into Doc (the data structure provided
 by the CCO library for pretty-printing), and finally makes use of CCO's rendering function
 to output the Html tree as a string. 
\par

\begin{haddockdesc}
\item[
mainFunc\ ::\ IO\ ()
]
\end{haddockdesc}
\begin{haddockdesc}
\item[\begin{tabular}{@{}l}
pipeline\ ::\ Component\ String\ String
\end{tabular}]\haddockbegindoc
the pipeline for pp-html. Read input, fold over resulting Html tree, and 
 output with pretty-print built-in functions. 
\par

\end{haddockdesc}
\begin{haddockdesc}
\item[\begin{tabular}{@{}l}
aTerm2Html\ ::\ Component\ ATerm\ Html
\end{tabular}]\haddockbegindoc
read an ATerm and return Html (abstract tree)
\par

\end{haddockdesc}
\begin{haddockdesc}
\item[\begin{tabular}{@{}l}
html2String\ ::\ Component\ Html\ String
\end{tabular}]\haddockbegindoc
pretty print the Html tree. Do this using html2ppAlgebra, an algebra to fold 
 over the Html with. 
\par

\end{haddockdesc}
\begin{haddockdesc}
\item[\begin{tabular}{@{}l}
html2ppAlgebra\ ::\ HtmlAlgebra\ Doc\ Doc\ Doc\ Doc\ Doc
\end{tabular}]\haddockbegindoc
the meat of this module. Defines an algebra that when used to fold over an Html 
 structure, yields a Doc, which can be pretty-printed. The way this is done, is that 
 for each possibly type found in Html, we define how to pretty-print it.
\par

\end{haddockdesc}
\begin{haddockdesc}
\item[\begin{tabular}{@{}l}
tagWithFields
\end{tabular}]\haddockbegindoc
\haddockbeginargs
\haddockdecltt{::} & \haddockdecltt{[Field]} & attribute list, pretty-printed with printField
 \\
                                               \haddockdecltt{->} & \haddockdecltt{String} & the tag name, A for example
 \\
                                                                                             \haddockdecltt{->} & \haddockdecltt{Doc} & output pretty-print CCO format
 \\
\end{tabulary}\par
helper function to create an html tag given the element name and a list of attributes.
\par

\end{haddockdesc}
\begin{haddockdesc}
\item[\begin{tabular}{@{}l}
printField\ ::\ Field\ ->\ Doc
\end{tabular}]\haddockbegindoc
used to print an attribute of an Html element. Simple, key=\haddocktt{value}. 
\par

\end{haddockdesc}