\haddockmoduleheading{Common.BibTypes}
\label{module:Common.BibTypes}
\haddockbeginheader
{\haddockverb\begin{verbatim}
module Common.BibTypes (
    BibTex(BibTex),  Entry(Entry, entryType, reference, fields),  Field(Field), 
    getKey,  getValue,  maybegetKey,  maybegetValue,  EntryType,  Reference, 
    BibTexAlgebra,  foldBibTex,  lookupField,  compareF
  ) where\end{verbatim}}
\haddockendheader

The decision has been made to represent a BibTeX file as a list of entries, along with possibly some preamble. 
\par

\begin{haddockdesc}
\item[\begin{tabular}{@{}l}
data\ BibTex
\end{tabular}]\haddockbegindoc
\haddockbeginconstrs
\haddockdecltt{=} & \haddockdecltt{BibTex [String] [Entry]} & \\
\end{tabulary}\par
a bibtex file is just a list of entries, with a list of preamble-strings
\par

\end{haddockdesc}
\begin{haddockdesc}
\item[\begin{tabular}{@{}l}
instance\ Show\ BibTex\\instance\ Tree\ BibTex
\end{tabular}]
\end{haddockdesc}
\begin{haddockdesc}
\item[\begin{tabular}{@{}l}
data\ Entry
\end{tabular}]\haddockbegindoc
\haddockbeginconstrs
\haddockdecltt{=} & \haddockdecltt{Entry} & \\
                    \haddockdecltt{entryType :: EntryType} & book, thesis, etc.

                    \haddockdecltt{reference :: Reference} & the name 

                    \haddockdecltt{fields :: [Field]} & a list of key/value pairs

\end{tabulary}\par
a bibtex entry has a type, a reference (it's name), and a list of fields
\par

\end{haddockdesc}
\begin{haddockdesc}
\item[\begin{tabular}{@{}l}
instance\ Eq\ Entry
\end{tabular}]\haddockbegindoc
We implement equality on entries. When their names are the same, we consider them equal.
\par

\end{haddockdesc}
\begin{haddockdesc}
\item[\begin{tabular}{@{}l}
instance\ Show\ Entry\\instance\ Tree\ Entry
\end{tabular}]
\end{haddockdesc}
\begin{haddockdesc}
\item[\begin{tabular}{@{}l}
data\ Field
\end{tabular}]\haddockbegindoc
\haddockbeginconstrs
\haddockdecltt{=} & \haddockdecltt{Field String String} & \\
\end{tabulary}\par
a field is an attibute/value pair
\par

\end{haddockdesc}
\begin{haddockdesc}
\item[\begin{tabular}{@{}l}
instance\ Eq\ Field
\end{tabular}]\haddockbegindoc
Fields are considered equal when their keys are the same.
\par

\end{haddockdesc}
\begin{haddockdesc}
\item[\begin{tabular}{@{}l}
instance\ Ord\ Field\\instance\ Show\ Field\\instance\ Tree\ Field
\end{tabular}]
\end{haddockdesc}
\begin{haddockdesc}
\item[\begin{tabular}{@{}l}
getKey\ ::\ Field\ ->\ String
\end{tabular}]\haddockbegindoc
returns the key part, given a Field
\par

\end{haddockdesc}
\begin{haddockdesc}
\item[\begin{tabular}{@{}l}
getValue\ ::\ Field\ ->\ String
\end{tabular}]\haddockbegindoc
returns the value part, given a Field
\par

\end{haddockdesc}
\begin{haddockdesc}
\item[\begin{tabular}{@{}l}
maybegetKey\ ::\ Maybe\ Field\ ->\ Maybe\ String
\end{tabular}]\haddockbegindoc
sometimes we want to get the key from a Maybe Field. 
\par

\end{haddockdesc}
\begin{haddockdesc}
\item[\begin{tabular}{@{}l}
maybegetValue\ ::\ Maybe\ Field\ ->\ Maybe\ String
\end{tabular}]\haddockbegindoc
...and sometimes we want to get the value from a Maybe Field. 
\par

\end{haddockdesc}
\begin{haddockdesc}
\item[\begin{tabular}{@{}l}
type\ EntryType\ =\ String
\end{tabular}]\haddockbegindoc
the entry type is characterised by a string, for example \haddocktt{book}
\par

\end{haddockdesc}
\begin{haddockdesc}
\item[\begin{tabular}{@{}l}
type\ Reference\ =\ String
\end{tabular}]\haddockbegindoc
the reference is also just a string, such as \haddocktt{pierce02}
\par

\end{haddockdesc}
\begin{haddockdesc}
\item[\begin{tabular}{@{}l}
type\ BibTexAlgebra\ bibtex\ preamble\ entry\ =\ ({\char 91}preamble{\char 93}\\\ \ \ \ \ \ \ \ \ \ \ \ \ \ \ \ \ \ \ \ \ \ \ \ \ \ \ \ \ \ \ \ \ \ \ \ \ \ \ \ \ \ \ \ ->\ {\char 91}entry{\char 93}\ ->\ bibtex,\ String\ ->\ preamble,\ EntryType\\\ \ \ \ \ \ \ \ \ \ \ \ \ \ \ \ \ \ \ \ \ \ \ \ \ \ \ \ \ \ \ \ \ \ \ \ \ \ \ \ \ \ \ \ \ \ \ \ \ \ \ \ \ \ \ \ \ \ \ \ \ \ \ \ \ \ \ \ \ \ \ \ \ \ \ \ \ \ \ \ \ \ \ \ \ \ ->\ Reference\\\ \ \ \ \ \ \ \ \ \ \ \ \ \ \ \ \ \ \ \ \ \ \ \ \ \ \ \ \ \ \ \ \ \ \ \ \ \ \ \ \ \ \ \ \ \ \ \ \ \ \ \ \ \ \ \ \ \ \ \ \ \ \ \ \ \ \ \ \ \ \ \ \ \ \ \ \ \ \ \ \ \ \ \ \ \ \ \ \ ->\ {\char 91}Field{\char 93}\\\ \ \ \ \ \ \ \ \ \ \ \ \ \ \ \ \ \ \ \ \ \ \ \ \ \ \ \ \ \ \ \ \ \ \ \ \ \ \ \ \ \ \ \ \ \ \ \ \ \ \ \ \ \ \ \ \ \ \ \ \ \ \ \ \ \ \ \ \ \ \ \ \ \ \ \ \ \ \ \ \ \ \ \ \ \ \ \ \ \ \ \ ->\ entry)
\end{tabular}]\haddockbegindoc
the type of the bibtex algebra. Used to fold over a bibtex library,
 like when we want to convert a BibTeX structure into HTML. 
\par
This seems the most natural way to define possible conversions from BibTex to 
 other (possibly tree-like) formats, such as Html later on. 
\par

\end{haddockdesc}
\begin{haddockdesc}
\item[\begin{tabular}{@{}l}
foldBibTex\ ::\ BibTexAlgebra\ bibtex\ preamble\ entry\\\ \ \ \ \ \ \ \ \ \ \ \ \ \ ->\ BibTex\ ->\ bibtex
\end{tabular}]\haddockbegindoc
How to fold over a BibTeX tree. Used when converting to Html in Bib2HTML.Tool.
\par

\end{haddockdesc}
\begin{haddockdesc}
\item[\begin{tabular}{@{}l}
lookupField\ ::\ String\ ->\ {\char 91}Field{\char 93}\ ->\ Maybe\ Field
\end{tabular}]\haddockbegindoc
Given a key, find the corresponding Field in a list of Fields. If it can't be found, Nothing is returned. 
\par

\end{haddockdesc}
\begin{haddockdesc}
\item[\begin{tabular}{@{}l}
compareF\ ::\ Field\ ->\ Field\ ->\ Ordering
\end{tabular}]\haddockbegindoc
Our implementation of ordering on Fields. This is how we make sure
 that author, then title, then the other fields, and finally year, are 
 displayed, regardless of how they are placed in the .bib file. This implementation
 allows us to simply run \haddocktt{sort} on a list of Fields. 
\par

\end{haddockdesc}